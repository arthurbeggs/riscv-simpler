%% LyX 2.1.4 created this file.  For more info, see http://www.lyx.org/.
%% Do not edit unless you really know what you are doing.
\documentclass[a4paper,oneside,brazil,11pt,a4paper,openright,titlepage,usenames,dvipsnames]{book}
\usepackage[T1]{fontenc}
\usepackage[latin9]{inputenc}
\setcounter{secnumdepth}{3}
\setcounter{tocdepth}{3}
\usepackage{array}
\usepackage{verbatim}
\usepackage{calc}
\usepackage{textcomp}
\usepackage{amssymb}
\usepackage{graphicx}

\makeatletter

%%%%%%%%%%%%%%%%%%%%%%%%%%%%%% LyX specific LaTeX commands.
\pdfpageheight\paperheight
\pdfpagewidth\paperwidth

%% Because html converters don't know tabularnewline
\providecommand{\tabularnewline}{\\}

%%%%%%%%%%%%%%%%%%%%%%%%%%%%%% User specified LaTeX commands.
% Classe alternativa, apropriada para impressão frente-verso. Inclui páginas em branco
% de forma que capítulos sempre tenham início na página à direita:
% \documentclass[11pt,a4paper,openright,titlepage]{book}

% Pacotes
\usepackage[T1]{fontenc}
\usepackage[brazilian]{babel}
\usepackage{epsfig}
\usepackage{subfigure}
\usepackage{amsfonts}
\usepackage{amsmath}
\usepackage[thmmarks,amsmath]{ntheorem}%\usepackage{amsthm}
\usepackage{boxedminipage}
\usepackage{geometry}
\usepackage{theorem}
\usepackage{fancybox}
\usepackage{fancyhdr}
\usepackage{ifthen}
\usepackage{url}
\usepackage{afterpage}
\usepackage{color}
\usepackage{colortbl}
\usepackage{rotating}
\usepackage{makeidx}
\usepackage{indentfirst}
% Pacotes para adição de figuras do inkscape
\usepackage{graphicx}
\usepackage{import}

% Escolher um dos seguintes formatos:
\usepackage{ft2unb} % segue padrão de fontes do Latex

\makeindex

\makeatother

\usepackage{babel}
\begin{document}
\setcounter{secnumdepth}{3}
\setcounter{tocdepth}{2}
\pagestyle{empty}

\grau{Engenheiro de Controle e Automação}

\tipodemonografia{TRABALHO DE GRADUAÇÃO}

\begin{comment}
Título
\end{comment}


\titulolinhai{TÍTULO DO TRABALHO DIVIDIDO EM}

\titulolinhaii{MAIS DE UMA LINHA PARA TÍTULOS}

\titulolinhaiii{REALMENTE LONGOS COMO ESTE}

\titulolinhaiv{}

\begin{comment}
Autores. Basta retirar o texto totalmente caso não haja um determinado
autor.
\end{comment}


\autori{João da Silva}

\autorii{}

\autoriii{}

\begin{comment}
Membros da banca. Basta retirar o texto totalmente caso não haja um
determinado membro da banca.
\end{comment}


\membrodabancai{Prof. Fulano de Sousa, ENE/UnB}

\membrodabancaifuncao{Orientador}

\membrodabancaii{Prof. Ciclano dos Santos, ENE/UnB}

\membrodabancaiifuncao{Examinador interno}

\membrodabancaiii{Prof. Beltrano da Silva, ENE/UnB}

\membrodabancaiiifuncao{Examinador interno}

\membrodabancaiv{}

\membrodabancaivfuncao{}

\membrodabancav{}

\membrodabancavfuncao{}

\begin{comment}
Data de defesa: mês e ano
\end{comment}


\mes{dezembro}
\ano{2015}

\begin{comment}
Comandos para criar a capa e a página de assinaturas
\end{comment}


\capaprincipal
\capaassinaturas

\begin{comment}
Ficha Catalográfica
\end{comment}


\noindent \textbf{FICHA CATALOGRÁFICA}

\noindent %
\fbox{\begin{minipage}[t]{1\columnwidth}%
JOÃO, DA SILVA

Título do trabalho dividido em mais de uma linha para títulos realmente
longos como este,

\medskip{}


{[}Distrito Federal{]} 2015.

\medskip{}


x, 101p., 297 mm (FT/UnB, Engenheiro, Controle e Automação, 2015).
Trabalho de Graduação \textendash{} Universidade de Brasília.Faculdade
de Tecnologia.

\medskip{}


1. Bla\hfill{}2.Ble\hfill{}

3. Bli

\medskip{}


I. Mecatrônica/FT/UnB\hfill{}II. Título (Série)\hfill{}

%
\end{minipage}}

\noindent \medskip{}


\noindent \textbf{REFERÊNCIA BIBLIOGRÁFICA}

SILVA, JOÃO DA, (2015). Título do trabalho dividido em mais de uma
linha para títulos realmente longos como este. Trabalho de Graduação
em Engenharia de Controle e Automação, Publicação FT.TG-$n^{\circ}022$,
Faculdade de Tecnologia, Universidade de Brasília, Brasília, DF, 101p.

\noindent \bigskip{}


\noindent \textbf{CESSÃO DE DIREITOS}

\noindent AUTOR: João da Silva

TÍTULO DO TRABALHO DE GRADUAÇÃO: Título do trabalho dividido em mais
de uma linha para títulos realmente longos como este.

\noindent \medskip{}


\noindent GRAU: Engenheiro\hfill{}ANO: 2015\hfill{}

\noindent \medskip{}


É concedida à Universidade de Brasília permissão para reproduzir cópias
deste Trabalho de Graduação e para emprestar ou vender tais cópias
somente para propósitos acadêmicos e científicos. O autor reserva
outros direitos de publicação e nenhuma parte desse Trabalho de Graduação
pode ser reproduzida sem autorização por escrito do autor.

\noindent \bigskip{}


\noindent \rule[0.5ex]{1\columnwidth}{1pt}

\noindent João da Silva

\noindent Rua dos Bobos, nº 0, Bairro Feliz.

\noindent 71000-000 Brasília \textendash{} DF \textendash{} Brasil.


\begin{comment}
Dedicatória
\end{comment}


\frontmatter

\begin{comment}
Texto de dedicatória do primeiro autor.
\end{comment}


\dedicatoriaautori{um beijo pra minha mãe, pro meu pai, e pra você}

\begin{comment}
Texto de dedicatória do segundo autor. Caso não tenha um segundo autor,
este texto não será mostrado
\end{comment}


\dedicatoriaautorii{Dedicatória do autor 2}

\begin{comment}
Texto de dedicatória do terceiro autor. Caso não tenha um terceiro
autor, este texto não será mostrado
\end{comment}


\dedicatoriaautoriii{Dedicatória do autor 3}

\begin{comment}
Comando para criar a página de dedicatória
\end{comment}


\dedicatoria

\begin{comment}
Agradecimentos
\end{comment}


\begin{comment}
Texto de agradecimentos do primeiro autor.
\end{comment}


\agradecimentosautori{Agradecimentos!}

\begin{comment}
Texto de agradecimentos do segundo autor. Caso não tenha um segundo
autor, este texto não será mostrado.
\end{comment}


\agradecimentosautorii{A inclusão desta seção de agradecimentos é
opcional e fica à critério do(s) autor(es), que caso deseje(em) inclui-la
deverá(ão) utilizar este espaço, seguindo esta formatação.}

\begin{comment}
Texto de agradecimentos do terceiro autor. Caso não tenha um terceiro
autor, este texto não será mostrado.
\end{comment}


\agradecimentosautoriii{A inclusão desta seção de agradecimentos
é opcional e fica à critério do(s) autor(es), que caso deseje(em)
inclui-la deverá(ão) utilizar este espaço, seguindo esta formatação.}

\begin{comment}
Comando para criar a página de agradecimentos
\end{comment}


\agradecimentos

\resumo{resumo}{Resumo!

\medskip{}


Palavras Chave: bla, ble, bli

}\vspace*{2cm}


\resumo{Abstract}{Abstract, in English ofc!

\medskip{}


Keywords: bla, ble, bli

}

\begin{comment}
Listas de conteúdo, figuras e tabelas
\end{comment}


\sumario
\listadefiguras
\listadetabelas

\begin{comment}
Lista de Símbolos
\end{comment}


%TCIDATA{LaTeXparent=0,0,these.tex}


%\chapter*{\setfontarial\mdseries LISTA DE SÍMBOLOS} % se usar ft1unb.sty, descomente esta linha



\chapter*{LISTA DE SÍMBOLOS}

% se usar ft2unb.sty, descomente esta linha



% \subsection*{Símbolos Latinos}

% \begin{tabular}{p{0.1\textwidth}p{0.63\textwidth}>{\PreserveBacklash\raggedleft}p{0.15\textwidth}}
% $v$  & Velocidade linear  & {[}m/s{]}\tabularnewline
% \end{tabular}


% \subsection*{Símbolos Gregos}

% \begin{tabular}{p{0.1\textwidth}p{0.63\textwidth}>{\PreserveBacklash\raggedleft}p{0.15\textwidth}}
% $\omega$ & Velocidade angular & {[}rad/s{]}\tabularnewline
% \end{tabular}


% \subsection*{Grupos Adimensionais}
%
% \begin{tabular}{p{0.1\textwidth}p{0.8\textwidth}}
% i, k & Contador\tabularnewline
% \end{tabular}


% \subsection*{Subscritos}

% \begin{tabular}{p{0.1\textwidth}p{0.8\textwidth}}
% $ref$  & referência \tabularnewline
% $fer$  & ferramenta \tabularnewline
% $sis$  & sistema \tabularnewline
% $des$  & desejado\tabularnewline
% \end{tabular}


% \subsection*{Sobrescritos}

% \begin{tabular}{p{0.1\textwidth}p{0.8\textwidth}}
% $\cdot$  & Variação temporal \tabularnewline
% $-$  & Valor médio \tabularnewline
% \end{tabular}


\subsection*{Siglas}

\begin{tabular}{p{0.1\textwidth}p{0.8\textwidth}}
    {BSD} & {Distribuição de Software de Berkeley --- \textit{Berkeley Software Distribution}}\tabularnewline{}
    {CSR} & {Registradores de Controle e Estado --- \textit{Control and Status Registers}} \tabularnewline{}
    {FPGA} & {Arranjo de Portas Programáveis em Campo --- \textit{Field Programmable Gate Array}} \tabularnewline{}
    {hart} & {\textit{hardware thread}} \tabularnewline{}
    {ISA} & {Arquitetura do Conjunto de Instruções --- \textit{Instruction Set Architecture}} \tabularnewline{}
    {MIPS} & {Microprocessador sem Estágios Intertravados de \textit{Pipeline} --- \textit{Microprocessor without Interlocked Pipeline Stages}} \tabularnewline{}
    {OAC} & {Organização e Arquitetura de Computadores} \tabularnewline{}
    {RISC} & {Computador com Conjunto de Instruções Reduzido --- \textit{Reduced Instruction Set Computer}} \tabularnewline{}
    {SiMPLE} & {Ambiente de Aprendizado Uniciclo, Multiciclo e \textit{Pipeline} --- \textit{Single-cycle Multicycle Pipeline Learning Environment}} \tabularnewline{}
    {RAS} & {Pilha de Endereços de Retorno --- \textit{Return Address Stack}} \tabularnewline{}
     % &  \tabularnewline
     % &  \tabularnewline
     % &  \tabularnewline
\end{tabular}


\begin{comment}
Corpo Principal
\end{comment}


\mainmatter
\setcounter{page}{1}
\pagenumbering{arabic}
\pagestyle{plain}

\begin{comment}
Introdução
\end{comment}
\chapter{Introdução}\label{CapIntro}

% Resumo opcional. Comentar se não usar.
%\resumodocapitulo{Resumo opcional}


\section{Motivação}

    {O mercado de trabalho está a cada dia mais exigente, sempre buscando profissionais que conheçam as melhores e mais recentes ferramentas disponíveis. Além disso, muitos universitários se sentem desestimulados ao estudarem assuntos desatualizados e com baixa possibilidade de aproveitamento do conteúdo no mercado de trabalho. Isso alimenta o desinteresse pelos temas abordados e, em muitos casos, leva à evasão escolar. Assim, é importante renovar as matérias com novas tecnologias e tendências de mercado sempre que possível, a fim de instigar o interesse dos discentes e formar profissionais mais capacitados e preparados para as demandas da atualidade.}

    {Hoje, a disciplina de Organização e Arquitetura de Computadores da Universidade de Brasília é ministrada utilizando a arquitetura \textit{MIPS32}. Apesar da arquitetura \textit{MIPS32} ainda ter grande força no meio acadêmico (em boa parte devido a sua simplicidade e extensa bibliografia), sua aplicação na indústria tem diminuído consideravelmente na última década.}

    {Embora a curva de aprendizagem de linguagens \textit{Assembly} de alguns processadores \textit{RISC} seja relativamente baixa para quem já conhece o \textit{Assembly MIPS32}, aprender uma arquitetura atual traz o benefício de conhecer o \textit{estado da arte} da organização e arquitetura de computadores.}

    {Para a proposta de modernização da disciplina, foi escolhida a \textit{ISA RISC-V} desenvolvida na Divisão de Ciência da Computação da Universidade da Califórnia, Berkeley como substituta à \textit{ISA MIPS32}.}


\section{Por que \textit{RISC-V}?}

    {A \textit{ISA RISC-V} (lê-se \textit{``risk-five''}) é uma arquitetura \textit{open source} com licença \textit{BSD}, o que permite o seu livre uso para quaisquer fins, sem distinção de se o trabalho possui código-fonte aberto ou proprietário. Tal característica possibilita que grandes fabricantes utilizem a arquitetura para criar seus produtos, mantendo a proteção de propriedade intelectual sobre seus métodos de implementação e quaisquer subconjuntos de instruções não-\textit{standard} que as empresas venham a desenvolver, o que estimula investimentos em pesquisa e desenvolvimento.}

    {Empresas como Google, IBM, AMD, Nvidia, Hewlett Packard, Microsoft, Oracle, Qualcomm e Western Digital são algumas das fundadoras e investidoras da \textit{RISC-V Foundation}, órgão responsável pela governança da arquitetura. Isso demonstra o interesse das gigantes do mercado no sucesso e disseminação da arquitetura.}

    {A licença também permite que qualquer indivíduo produza, distribua e até mesmo comercialize sua própria implementação da arquitetura sem ter que arcar com \textit{royalties}, sendo ideal para pesquisas acadêmicas, \textit{startups} e até mesmo \textit{hobbyistas}.}

    {O conjunto de instruções foi desenvolvido tendo em mente seu uso em diversas escalas: sistemas embarcados, \textit{smartphones}, computadores pessoais, servidores e supercomputadores, o que permitirá maior reuso de \textit{software} e maior integração de \textit{hardware}.}

    {Outro fator que estimula o uso do \textit{RISC-V} é a modernização dos livros didáticos. A nova versão do livro utilizado em OAC, Organização e Projeto de Computadores, de David Patterson e John Hennessy, utiliza a \textit{ISA RISC-V}.}

    {Além disso, com a promessa de se tornar uma das arquiteturas mais utilizadas nos próximos anos, utilizar o \textit{RISC-V} como arquitetura da disciplina de OAC se mostra a escolha ideal no momento.}


\section{O Projeto \textit{RISC-V SiMPLE}}

    {O projeto \textit{RISC-V SiMPLE (Single-cycle Multicycle Pipeline Learning Environment)} consiste no desenvolvimento de um processador com conjunto de instruções \textit{RISC-V}, sintetizável em \textit{FPGA} e com \textit{hardware} descrito em \textit{Verilog}. A microarquitetura implementada nesse trabalho é uniciclo, escalar, em ordem, com um único \textit{hart} e com caminho de dados de 64 bits. Trabalhos futuros poderão utilizar a estrutura altamente configurável e modularizada do projeto para desenvolver as versões em microarquiteturas multiciclo e \textit{pipeline}.}

    {O processador contém o conjunto de instruções I (para operações com inteiros, sendo o único módulo com implementação mandatória pela arquitetura) e as extensões \textit{standard} M (para multiplicação e divisão de inteiros) e F (para ponto flutuante com precisão simples conforme o padrão IEEE 754 com revisão de 2008). O projeto não implementa as extensões D (ponto flutuante de precisão dupla) e A (operações atômicas de sincronização), e com isso o \textit{soft core} desenvolvido não pode ser definido como de propósito geral, G (que deve conter os módulos I, M, A, F e D). Assim, pela nomenclatura da arquitetura, o processador desenvolvido é um \textit{RV64IMF}.}

    {O projeto contempla \textit{traps}, interrupções, exceções, \textit{CSRs}, chamadas de sistema e outras funcionalidades de nível privilegiado da arquitetura.}

    {O \textit{soft core} possui barramento Avalon para se comunicar com os periféricos das plataformas de desenvolvimento. O projeto foi desenvolvido utilizando a placa DE2--115 com \textit{FPGA Altera Cyclone} e permite a fácil adaptação para outras placas da Altera.}


\begin{comment}
Fundamentos
\end{comment}

\chapter{Fundamentos\label{chap:FundamentacaoMatematica}}

% Resumo opcional. Comentar se não usar.
\resumodocapitulo{Resumo opcional.}


\section{Introdução}

Introduzir.


\begin{comment}
Conclusões
\end{comment}

\chapter{Conclus�es}

\label{CapConclusoes}

Concluir


\section{Perspectivas Futuras}

Perspectivas futuras


\begin{comment}
Bibliografia
\end{comment}


\renewcommand{\bibname}{REFERÊNCIAS BIBLIOGRÁFICAS}
\addcontentsline{toc}{chapter}{REFERÊNCIAS BIBLIOGRÁFICAS}

\bibliographystyle{abnt-num}
\bibliography{bibliography}


\begin{comment}
Anexos
\end{comment}


\anexos
\makeatletter
% não retirar estes comandos
\renewcommand{\@makechapterhead}[1]{%
  {\parindent \z@ \raggedleft \setfontarial\bfseries
\LARGE \thechapter. \space\space
\uppercase{#1}\par
\vskip 40\p@
}
}
\makeatother

\begin{comment}
Anexo I: Descrição do CD
\end{comment}



\chapter{Descrição do conteúdo do CD}

\label{AnCD}

Descrever CD.


\refstepcounter{noAnexo}

\begin{comment}
Anexo II: Programas Utilizados
\end{comment}



\chapter{Programas utilizados}

Quais programas foram utilizados?


\refstepcounter{noAnexo}

\begin{comment}
Acrescente mais anexos conforme julgar necessário.
\end{comment}

\end{document}
