\documentclass[a4paper,oneside,brazil,11pt,a4paper,openright,titlepage,usenames,dvipsnames]{book}
% Classe alternativa, apropriada para impressão frente-verso. Inclui páginas em branco de forma que capítulos sempre tenham início na página à direita:
% \documentclass[11pt,a4paper,openright,titlepage]{book}

\usepackage[utf8]{inputenc}
\usepackage[T1]{fontenc}
\usepackage[brazilian]{babel}
\usepackage{lmodern}
\usepackage{array}
\usepackage{verbatim}
\usepackage{calc}
\usepackage{textcomp}
\usepackage{gensymb}
\usepackage{amsfonts}
\usepackage{amsmath}
\usepackage[thmmarks,amsmath]{ntheorem}%\usepackage{amsthm}
\usepackage{amssymb}
\usepackage{graphicx}
\usepackage{float}
\usepackage[]{subfigure}
\usepackage{epsfig}
\usepackage{boxedminipage}
\usepackage{geometry}
\usepackage{theorem}
\usepackage{fancybox}
\usepackage{fancyhdr}
\usepackage{ifthen}
\usepackage{url}
\usepackage{afterpage}
\usepackage{color}
\usepackage{colortbl}
\usepackage{rotating}
\usepackage{makeidx}
\usepackage{indentfirst}
\usepackage{import}
\usepackage{enumitem}

% Escolher um dos seguintes formatos: (ft1unb) ou (ft2unb)
\usepackage{ft2unb} % ft2unb segue padrão de fontes do LaTeX;

\makeindex

\makeatother

\begin{document}
\setcounter{secnumdepth}{3}
\setcounter{tocdepth}{2}
\pagestyle{empty}

\grau{Engenheiro de Controle e Automação}

\tipodemonografia{TRABALHO DE GRADUAÇÃO}

% Título
\titulolinhai{RISC-V SiMPLE}
\titulolinhaii{}
\titulolinhaiii{}
\titulolinhaiv{}


% Autores
\autori{Arthur de Matos Beggs}
\autorii{}
\autoriii{}


% Membros da banca
\membrodabancai{Prof.\ Marcus Vinicius Lamar, CIC/UnB}
\membrodabancaifuncao{Orientador}

\membrodabancaii{Prof.\ Ricardo Pezzuol Jacobi, CIC/UnB}
\membrodabancaiifuncao{Co-Orientador}

\membrodabancaiii{}
\membrodabancaiiifuncao{}

\membrodabancaiv{}
\membrodabancaivfuncao{}

\membrodabancav{}
\membrodabancavfuncao{}


% Data de defesa
\mes{Dezembro}
\ano{2018}


% Comandos para criar a capa e a página de assinaturas
\capaprincipal{}
\capaassinaturas{}


% Ficha Catalográfica
\noindent \textbf{FICHA CATALOGRÁFICA}

\noindent %
\fbox{\begin{minipage}[t]{1\columnwidth}%
JOÃO, DA SILVA

Título do trabalho dividido em mais de uma linha para títulos realmente
longos como este,

\medskip{}


{[}Distrito Federal{]} 2015.

\medskip{}


x, 101p., 297 mm (FT/UnB, Engenheiro, Controle e Automação, 2015).
Trabalho de Graduação \textendash{} Universidade de Brasília.Faculdade
de Tecnologia.

\medskip{}


1. Bla\hfill{}2.Ble\hfill{}

3. Bli

\medskip{}


I. Mecatrônica/FT/UnB\hfill{}II. Título (Série)\hfill{}

%
\end{minipage}}

\noindent \medskip{}


\noindent \textbf{REFERÊNCIA BIBLIOGRÁFICA}

SILVA, JOÃO DA, (2015). Título do trabalho dividido em mais de uma
linha para títulos realmente longos como este. Trabalho de Graduação
em Engenharia de Controle e Automação, Publicação FT.TG-$n^{\circ}022$,
Faculdade de Tecnologia, Universidade de Brasília, Brasília, DF, 101p.

\noindent \bigskip{}


\noindent \textbf{CESSÃO DE DIREITOS}

\noindent AUTOR: João da Silva

TÍTULO DO TRABALHO DE GRADUAÇÃO: Título do trabalho dividido em mais
de uma linha para títulos realmente longos como este.

\noindent \medskip{}


\noindent GRAU: Engenheiro\hfill{}ANO: 2015\hfill{}

\noindent \medskip{}


É concedida à Universidade de Brasília permissão para reproduzir cópias
deste Trabalho de Graduação e para emprestar ou vender tais cópias
somente para propósitos acadêmicos e científicos. O autor reserva
outros direitos de publicação e nenhuma parte desse Trabalho de Graduação
pode ser reproduzida sem autorização por escrito do autor.

\noindent \bigskip{}


\noindent \rule[0.5ex]{1\columnwidth}{1pt}

\noindent João da Silva

\noindent Rua dos Bobos, nº 0, Bairro Feliz.

\noindent 71000-000 Brasília \textendash{} DF \textendash{} Brasil.



% Dedicatória
\frontmatter

\dedicatoriaautori{Dedico ao pato de borracha especialista em TI que sempre me ajuda a depurar meus códigos.}
\dedicatoriaautorii{}
\dedicatoriaautoriii{}

\dedicatoria{}


% Agradecimentos
\agradecimentosautori{Agradecimentos!}
\agradecimentosautorii{}
\agradecimentosautoriii{}

\agradecimentos{}


\resumo{resumo}{Resumo!

\medskip{}


Palavras Chave: RISC-V

}\vspace*{2cm}


\resumo{Abstract}{Abstract!

\medskip{}


Keywords: RISC-V

}

% Listas de conteúdo, figuras e tabelas
\sumario{}
\listadefiguras{}
\listadetabelas{}


% Lista de Símbolos
%TCIDATA{LaTeXparent=0,0,these.tex}


%\chapter*{\setfontarial\mdseries LISTA DE SÍMBOLOS} % se usar ft1unb.sty, descomente esta linha



\chapter*{LISTA DE SÍMBOLOS}

% se usar ft2unb.sty, descomente esta linha



% \subsection*{Símbolos Latinos}

% \begin{tabular}{p{0.1\textwidth}p{0.63\textwidth}>{\PreserveBacklash\raggedleft}p{0.15\textwidth}}
% $v$  & Velocidade linear  & {[}m/s{]}\tabularnewline
% \end{tabular}


% \subsection*{Símbolos Gregos}

% \begin{tabular}{p{0.1\textwidth}p{0.63\textwidth}>{\PreserveBacklash\raggedleft}p{0.15\textwidth}}
% $\omega$ & Velocidade angular & {[}rad/s{]}\tabularnewline
% \end{tabular}


% \subsection*{Grupos Adimensionais}
%
% \begin{tabular}{p{0.1\textwidth}p{0.8\textwidth}}
% i, k & Contador\tabularnewline
% \end{tabular}


% \subsection*{Subscritos}

% \begin{tabular}{p{0.1\textwidth}p{0.8\textwidth}}
% $ref$  & referência \tabularnewline
% $fer$  & ferramenta \tabularnewline
% $sis$  & sistema \tabularnewline
% $des$  & desejado\tabularnewline
% \end{tabular}


% \subsection*{Sobrescritos}

% \begin{tabular}{p{0.1\textwidth}p{0.8\textwidth}}
% $\cdot$  & Variação temporal \tabularnewline
% $-$  & Valor médio \tabularnewline
% \end{tabular}


\subsection*{Siglas}

\begin{tabular}{p{0.1\textwidth}p{0.8\textwidth}}
    {BSD} & {Distribuição de Software de Berkeley --- \textit{Berkeley Software Distribution}}\tabularnewline{}
    {CSR} & {Registradores de Controle e Estado --- \textit{Control and Status Registers}} \tabularnewline{}
    {FPGA} & {Arranjo de Portas Programáveis em Campo --- \textit{Field Programmable Gate Array}} \tabularnewline{}
    {hart} & {\textit{hardware thread}} \tabularnewline{}
    {ISA} & {Arquitetura do Conjunto de Instruções --- \textit{Instruction Set Architecture}} \tabularnewline{}
    {MIPS} & {Microprocessador sem Estágios Intertravados de \textit{Pipeline} --- \textit{Microprocessor without Interlocked Pipeline Stages}} \tabularnewline{}
    {OAC} & {Organização e Arquitetura de Computadores} \tabularnewline{}
    {RISC} & {Computador com Conjunto de Instruções Reduzido --- \textit{Reduced Instruction Set Computer}} \tabularnewline{}
    {SiMPLE} & {Ambiente de Aprendizado Uniciclo, Multiciclo e \textit{Pipeline} --- \textit{Single-cycle Multicycle Pipeline Learning Environment}} \tabularnewline{}
    {RAS} & {Pilha de Endereços de Retorno --- \textit{Return Address Stack}} \tabularnewline{}
     % &  \tabularnewline
     % &  \tabularnewline
     % &  \tabularnewline
\end{tabular}



% Corpo Principal
\mainmatter{}
\setcounter{page}{1}
\pagenumbering{arabic}
\pagestyle{plain}

\chapter{Introdução}\label{CapIntro}

% Resumo opcional. Comentar se não usar.
%\resumodocapitulo{Resumo opcional}


\section{Motivação}

    {O mercado de trabalho está a cada dia mais exigente, sempre buscando profissionais que conheçam as melhores e mais recentes ferramentas disponíveis. Além disso, muitos universitários se sentem desestimulados ao estudarem assuntos desatualizados e com baixa possibilidade de aproveitamento do conteúdo no mercado de trabalho. Isso alimenta o desinteresse pelos temas abordados e, em muitos casos, leva à evasão escolar. Assim, é importante renovar as matérias com novas tecnologias e tendências de mercado sempre que possível, a fim de instigar o interesse dos discentes e formar profissionais mais capacitados e preparados para as demandas da atualidade.}

    {Hoje, a disciplina de Organização e Arquitetura de Computadores da Universidade de Brasília é ministrada utilizando a arquitetura \textit{MIPS32}. Apesar da arquitetura \textit{MIPS32} ainda ter grande força no meio acadêmico (em boa parte devido a sua simplicidade e extensa bibliografia), sua aplicação na indústria tem diminuído consideravelmente na última década.}

    {Embora a curva de aprendizagem de linguagens \textit{Assembly} de alguns processadores \textit{RISC} seja relativamente baixa para quem já conhece o \textit{Assembly MIPS32}, aprender uma arquitetura atual traz o benefício de conhecer o \textit{estado da arte} da organização e arquitetura de computadores.}

    {Para a proposta de modernização da disciplina, foi escolhida a \textit{ISA RISC-V} desenvolvida na Divisão de Ciência da Computação da Universidade da Califórnia, Berkeley como substituta à \textit{ISA MIPS32}.}


\section{Por que \textit{RISC-V}?}

    {A \textit{ISA RISC-V} (lê-se \textit{``risk-five''}) é uma arquitetura \textit{open source} com licença \textit{BSD}, o que permite o seu livre uso para quaisquer fins, sem distinção de se o trabalho possui código-fonte aberto ou proprietário. Tal característica possibilita que grandes fabricantes utilizem a arquitetura para criar seus produtos, mantendo a proteção de propriedade intelectual sobre seus métodos de implementação e quaisquer subconjuntos de instruções não-\textit{standard} que as empresas venham a desenvolver, o que estimula investimentos em pesquisa e desenvolvimento.}

    {Empresas como Google, IBM, AMD, Nvidia, Hewlett Packard, Microsoft, Oracle, Qualcomm e Western Digital são algumas das fundadoras e investidoras da \textit{RISC-V Foundation}, órgão responsável pela governança da arquitetura. Isso demonstra o interesse das gigantes do mercado no sucesso e disseminação da arquitetura.}

    {A licença também permite que qualquer indivíduo produza, distribua e até mesmo comercialize sua própria implementação da arquitetura sem ter que arcar com \textit{royalties}, sendo ideal para pesquisas acadêmicas, \textit{startups} e até mesmo \textit{hobbyistas}.}

    {O conjunto de instruções foi desenvolvido tendo em mente seu uso em diversas escalas: sistemas embarcados, \textit{smartphones}, computadores pessoais, servidores e supercomputadores, o que permitirá maior reuso de \textit{software} e maior integração de \textit{hardware}.}

    {Outro fator que estimula o uso do \textit{RISC-V} é a modernização dos livros didáticos. A nova versão do livro utilizado em OAC, Organização e Projeto de Computadores, de David Patterson e John Hennessy, utiliza a \textit{ISA RISC-V}.}

    {Além disso, com a promessa de se tornar uma das arquiteturas mais utilizadas nos próximos anos, utilizar o \textit{RISC-V} como arquitetura da disciplina de OAC se mostra a escolha ideal no momento.}


\section{O Projeto \textit{RISC-V SiMPLE}}

    {O projeto \textit{RISC-V SiMPLE (Single-cycle Multicycle Pipeline Learning Environment)} consiste no desenvolvimento de um processador com conjunto de instruções \textit{RISC-V}, sintetizável em \textit{FPGA} e com \textit{hardware} descrito em \textit{Verilog}. A microarquitetura implementada nesse trabalho é uniciclo, escalar, em ordem, com um único \textit{hart} e com caminho de dados de 64 bits. Trabalhos futuros poderão utilizar a estrutura altamente configurável e modularizada do projeto para desenvolver as versões em microarquiteturas multiciclo e \textit{pipeline}.}

    {O processador contém o conjunto de instruções I (para operações com inteiros, sendo o único módulo com implementação mandatória pela arquitetura) e as extensões \textit{standard} M (para multiplicação e divisão de inteiros) e F (para ponto flutuante com precisão simples conforme o padrão IEEE 754 com revisão de 2008). O projeto não implementa as extensões D (ponto flutuante de precisão dupla) e A (operações atômicas de sincronização), e com isso o \textit{soft core} desenvolvido não pode ser definido como de propósito geral, G (que deve conter os módulos I, M, A, F e D). Assim, pela nomenclatura da arquitetura, o processador desenvolvido é um \textit{RV64IMF}.}

    {O projeto contempla \textit{traps}, interrupções, exceções, \textit{CSRs}, chamadas de sistema e outras funcionalidades de nível privilegiado da arquitetura.}

    {O \textit{soft core} possui barramento Avalon para se comunicar com os periféricos das plataformas de desenvolvimento. O projeto foi desenvolvido utilizando a placa DE2--115 com \textit{FPGA Altera Cyclone} e permite a fácil adaptação para outras placas da Altera.}

\chapter{A \textit{ISA RISC-V}}\label{CapISA}

% Resumo opcional. Comentar se não usar.
%\resumodocapitulo{Resumo opcional}

    \section{Visão Geral da Arquitetura}

        {A \textit{ISA RISC-V} é uma arquitetura modular, sendo o módulo base de operações com inteiros mandatório em qualquer implementação. Os demais módulos são extensões de uso opcional. A arquitetura não suporta \textit{branch delay slots} e aceita instruções de tamanho variável. A codificação das instruções de tamanho variável é mostrada na Figura~\ref{fig:riscv_var_length}. As instruções presentes no módulo base correspondem ao mínimo necessário para emular por \textit{software} as demais extensões (com exceção das operações atômicas).}

        \begin{figure}[H]
        \centering
            \includegraphics[width=1\linewidth]{figs/RV_InstructionLength.png}
            \caption{Codificação de instruções de tamanho variável da arquitetura \textit{RISC-V}.}\label{fig:riscv_var_length}
        \end{figure}

        \clearpage

        {A nomenclatura do conjunto de instruções implementado segue a seguinte estrutura:}

        \begin{itemize}[leftmargin=20mm]
            \item {As letras ``RV'';}
            \item {A largura dos registradores do módulo Inteiro;}
            \item {A letra ``I'' representando a base Inteira. Caso o subconjunto Embarcado (\textit{Embedded}) seja implementado, substitui-se pela letra ``E'';}
            \item {Demais letras identificadoras de módulos opcionais.}
        \end{itemize}

        {Assim, uma implementação com registradores de 64 bits somente com o módulo base de Inteiros é denominado ``RV64I''.}


        \section{Módulo Inteiro}

            {}


        \section{Extensões}

            \subsection{Extensão M}

                {}


            \subsection{Extensão A}

                {}


            \subsection{Extensão F}

                {}


            \subsection{Extensão D}

                {}


            \subsection{Outras Extensões}

                {}

    \section{Arquitetura Privilegiada}

        {}


    \section{Formatos de Instruções}

        \begin{figure}[H]
        \centering
            \includegraphics[width=1\linewidth]{figs/RV_Formats.png}
            \caption{Formatos de Instruções da \textit{ISA RISC-V}.}\label{fig:riscv_formats}
        \end{figure}


        % \begin{figure}[H]
        % \centering
        %     \includegraphics[width=1\linewidth]{figs/MIPS_Formats.png}
        %     \caption{Formatos de Instruções da \textit{ISA MIPS32}.}\label{fig:mips_formats}
        % \end{figure}


    \section{Formatos de Imediatos}

        \begin{figure}[H]
        \centering
            \includegraphics[width=1\linewidth]{figs/RV_I_Imm.png}
            \caption{Formação do Imediato de tipo I.}\label{fig:riscv_i_imm}
        \end{figure}

        \begin{figure}[H]
        \centering
            \includegraphics[width=1\linewidth]{figs/RV_S_Imm.png}
            \caption{Formação do Imediato de tipo S.}\label{fig:riscv_s_imm}
        \end{figure}

        \begin{figure}[H]
        \centering
            \includegraphics[width=1\linewidth]{figs/RV_B_Imm.png}
            \caption{Formação do Imediato de tipo B.}\label{fig:riscv_b_imm}
        \end{figure}

        \begin{figure}[H]
        \centering
            \includegraphics[width=1\linewidth]{figs/RV_U_Imm.png}
            \caption{Formação do Imediato de tipo U.}\label{fig:riscv_u_imm}
        \end{figure}

        \begin{figure}[H]
        \centering
            \includegraphics[width=1\linewidth]{figs/RV_J_Imm.png}
            \caption{Formação do Imediato de tipo J.}\label{fig:riscv_j_imm}
        \end{figure}


        % \begin{figure}[H]
        % \centering
        %     \includegraphics[width=1\linewidth]{figs/MIPS_Immediates.png}
        %     \caption{Formatos de Imediato da \textit{ISA MIPS32}.}\label{fig:mips_immediates}
        % \end{figure}

\chapter{Implementação}\label{CapImpl}

% Resumo opcional. Comentar se não usar.
%\resumodocapitulo{Resumo opcional}


    \section{Caminho de Dados}

        {O caminho de dados projetado para a implementação da microarquitetura uniciclo é apresentado na Figura~\ref{fig:datapath}.}

        \begin{figure}[H]
        \centering
            \includegraphics[width=1\linewidth]{figs/singlecycle.png}
            \caption{Caminho de Dados implementado para o módulo I.}\label{fig:datapath}
        \end{figure}

        {O \textit{datapath} possui um banco de 32 registradores de uso geral de 64 bits cada. A memória possui arquitetura Harvard, sendo a memória de instruções (\textit{text}) \textit{read-only} e a memória de dados (\textit{data}) \textit{read-write}. São implementadas 49 instruções, sendo elas:}

        \begin{itemize}[leftmargin=20mm]
            \item {LUI:\@ Load Upper Intermediate;}
            \item {AUIPC:\@ Add Upper Intermediate to Program Counter;}
            \item {JAL:\@ Jump And Link;}
            \item {JALR:\@ Jump And Link Register;}
            \item {BEQ:\@ Branch if EQual;}
            \item {BNE:\@ Branch if Not Equal;}
            \item {BLT:\@ Branch if Less Than;}
            \item {BGE:\@ Branch if Greater or Equal;}
            \item {BLTU:\@ Branch if Less Than Unsigned;}
            \item {BGEU:\@ Branch if Greater or Equal Unsigned;}
            \item {LB:\@ Load Byte;}
            \item {LH:\@ Load Halfword;}
            \item {LW:\@ Load Word;}
            \item {LBU:\@ Load Byte Unsigned;}
            \item {LHU:\@ Load Halfword Unsigned;}
            \item {SB:\@ Store Byte;}
            \item {SH:\@ Store Halfword;}
            \item {SW:\@ Store Word;}
            \item {ADDI:\@ ADD Immediate;}
            \item {SLTI:\@ Set on Less Than;}
            \item {SLTIU:\@ Set on Less Than Unsigned;}
            \item {XORI:\@ XOR Immediate;}
            \item {ORI:\@ OR Immediate;}
            \item {ANDI:\@ AND Immediate;}
            \item {SLLI:\@ Shift Left Logical Immedate;}
            \item {SRLI:\@ Shift Right Logical Immediate;}
            \item {SRAI:\@ Shift Right Arithmetic Immediate;}
            \item {ADD:\@ ADD;}
            \item {SUB:\@ SUB;}
            \item {SLL:\@ Shift Left Logical;}
            \item {SLT:\@ Set on Less Than;}
            \item {SLTU:\@ Set on Less Than Unsigned;}
            \item {XOR:\@ XOR;}
            \item {SRL:\@ Shift Right Logical;}
            \item {SRA:\@ Shift Right Arithmetic;}
            \item {OR:\@ OR;}
            \item {AND:\@ AND;}
            \item {LWU:\@ Load Word Unsigned;}
            \item {LD:\@ Load Double;}
            \item {SD:\@ Store Double;}
            \item {ADDIW:\@ ADD Immediate Word-size;}
            \item {SLLIW:\@ Shift Left Logical Immedate Word-size;}
            \item {SRLIW:\@ Shift Right Logical Immediate Word-size;}
            \item {SRAIW:\@ Shift Right Arithmetic Immediate Word-size;}
            \item {ADDW:\@ ADD Word-size;}
            \item {SUBW:\@ SUB Word-size;}
            \item {SLLW:\@ Shift Left Logical Word-size;}
            \item {SRLW:\@ Shift Right Logical Word-size;}
            \item {SRAW:\@ Shift Right Arithmetic Word-size;}
        \end{itemize}

        {Para que o processador seja completamente compatível com a especificação da \textit{ISA}, falta implementar tratamentos de exceções, interrupções e \textit{traps}, Registradores \textit{CSR}, instruções de chamada ao ambiente (ECALL/EBREAK), instruções de \textit{fencing} de memória, suporte ao acesso desalinhado à memória de dados e pilha de endereço de retorno (RAS).}

\chapter{Resultados}\label{CapResult}

% Resumo opcional. Comentar se não usar.
%\resumodocapitulo{Resumo opcional}


\chapter{Conclus�es}

\label{CapConclusoes}

Concluir


\section{Perspectivas Futuras}

Perspectivas futuras



% Bibliografia
\renewcommand{\bibname}{REFERÊNCIAS BIBLIOGRÁFICAS}
\addcontentsline{toc}{chapter}{REFERÊNCIAS BIBLIOGRÁFICAS}

\bibliographystyle{abnt-num}
\bibliography{bibliography}


% Anexos
\anexos{}
\makeatletter
% não retirar estes comandos
\renewcommand{\@makechapterhead}[1]{%
  {\parindent \z@ \raggedleft \setfontarial\bfseries
\LARGE \thechapter. \space\space
\uppercase{#1}\par
\vskip 40\p@
}
}
\makeatother


% Anexo I: Descrição do CD

\chapter{Descrição do conteúdo do CD}

\label{AnCD}

Descrever CD.


\refstepcounter{noAnexo}


% Anexo II: Programas Utilizados

\chapter{Programas utilizados}

Quais programas foram utilizados?


\refstepcounter{noAnexo}


% Acrescente mais anexos conforme julgar necessário.

\end{document}
