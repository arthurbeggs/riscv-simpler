\chapter{Especificação}

\label{CapSpecs}

% Resumo opcional. Comentar se não usar.
%\resumodocapitulo{Resumo opcional}

    {A proposta do projeto consiste no desenvolvimento de um processador sintetizável em \textit{FPGA}, utilizando o conjunto de instruções \textit{RISC-V}. O caminho da dados inicialmente implementado será um Uniciclo de 64 bits, já prevendo sua expansão para um \textit{datapath} Multiciclo e um \textit{Pipeline}.}

    {O processador deverá conter os subconjuntos de instruções I (para operações com inteiros, sendo o único subconjunto com implementação mandatória pela arquitetura), M (para multiplicação e divisão de inteiros), F (para ponto flutuante com precisão simples conforme o padrão IEEE 754 com revisão de 2008) e possivelmente D (ponto flutuante de precisão dupla). No presente momento, a implementação do subconjunto D é duvidosa e a do subconjunto A (operações atômicas de sincronização) está descartada, e com isso o trabalho desenvolvido não pode ser definido como de propósito geral, G (que deve conter os pacotes I, M, A, F e D). Assim, pela nomenclatura da arquitetura, o processador desenvolvido será um \textit{RV64IMF}.}

    {O projeto também pretende contemplar \textit{traps}, interrupções, exceções e outras funcionalidades de nível privilegiado da arquitetura. Porém, no presente momento as especificações do nível privilegiado da arquitetura encontram-se em versão \textit{draft}. Com isso, a definição do que será implementado da camada privilegiada deverá aguardar a publicação da versão \textit{standard}.}

    {Uma vez que o projeto utilizará a estrutura do processador \textit{MIPS-PUM} (processador de arquitetura \textit{MIPS32} atualmente desenvolvido na turma de OAC do prof. Marcus Vinicuis Lamar) como referência, as interfaces e controladores dos periféricos serão reaproveitadas, sendo reescritas e melhor documentadas quando possível e/ou necessário.}

    {Como a \textit{ISA RISC-V} é relativamente nova, faltam ferramentas didáticas (como o simulador da \textit{ISA MIPS}, o \textit{MARS}) para uma aplicabilidade adequada do conteúdo em sala de aula e em laboratório. Desta forma, é necessária a implementação de um \textit{IDE} (\textit{Integrated Development Environment}, ou Ambiente de Desenvolvimento Integrado) para a escrita de código \textit{Assembly RISC-V}, além de montagem e simulação do \textit{Assembly}. Uma outra possibilidade é a criação de um pacote com sintaxe, montador e simulador da arquitetura para ser utilizado em um \textit{IDE} já existente (e.g. Atom Editor).}
